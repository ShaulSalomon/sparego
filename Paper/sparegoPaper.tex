\documentclass[10pt]{llncs}
\pagestyle{plain}

\usepackage[cmex10]{amsmath}
\usepackage{amssymb}
\usepackage{array}
\usepackage{bm}
\usepackage{graphicx}
\usepackage{epstopdf}
\usepackage[hang]{subfigure}
\usepackage{import}
\usepackage{booktabs}
\usepackage{tabularx}

\usepackage[colorlinks=true,linkcolor=blue,citecolor=blue,urlcolor=black]{hyperref}
\usepackage{algorithm, algpseudocode}

% Commenting functionality
\usepackage{todonotes}
%\reversemarginpar

\newcommand{\brs}[1]{\left[{#1}\right]} %square bracets
\newcommand{\brr}[1]{{\left({#1}\right)}} %round bracets
\newcommand{\brf}[1]{\left\lbrace{#1}\right\rbrace} %figure bracets
\newcommand{\brabs}[1]{\left\vert{#1}\right\vert} %absolute bracets
\newcommand{\norm}[2]{\left\|{#1}\right\|_{#2}} %norm_*

\newcommand{\I}[1]{I\!\left[{#1}\right]} %Robustness criterion I[*]
\newcommand{\Iq}[1]{I_q\!\left[{#1}\right]} %Robustness criterion I_q[*]
\newcommand{\Ic}[1]{I_c\!\left[{#1}\right]} %Robustness criterion I_c[*]

\newcommand{\vx}{\mathbf{x}} % vector x
\newcommand{\vX}{\mathbf{X}} % vector X
\newcommand{\vf}{\mathbf{f}} % vector f
\newcommand{\vz}{\mathbf{z}} % vector z
\newcommand{\vZ}{\mathbf{Z}} % vector Z
\newcommand{\vw}{\mathbf{w}} % vector w
\newcommand{\vd}{\mathbf{d}} % vector d
\newcommand{\vp}{\mathbf{p}} % vector p
\newcommand{\vP}{\mathbf{P}} % vector P
\newcommand{\vu}{\mathbf{u}} % vector u
\newcommand{\vU}{\mathbf{U}} % vector U
\newcommand{\DSet}{\mathcal{D}} % Set D
\newcommand{\NSet}{\mathcal{N}} % Set N
\newcommand{\XSet}{\mathcal{X}} % set X
\newcommand{\ZSet}{\mathcal{Z}} % set Z
\newcommand{\SSet}{\mathcal{S}} % set S
\newcommand{\ISet}{\mathcal{I}} % set I
\newcommand{\Pref}{\mathcal{P}} % Preferences set P

\DeclareMathOperator*{\argmin}{\arg\!\min}
\DeclareMathOperator*{\argmax}{\arg\!\max}


\begin{document}
\title{sParEGO -- A Hybrid Optimization Algorithm for Expensive Uncertain Many-Objective Optimization Problems}
\author{Robin C. Purshouse\inst{1} \and Shaul Salomon\inst{2} \and Daniel C. Oara\inst{1} \and Peter J. Fleming\inst{1}}
\institute{Department of Automatic Control and Systems Engineering\\ University of Sheffield\\ Mappin Street, Sheffield S1 3JD, UK\\ \email{\{r.purshouse,dcoara1,p.fleming\}@sheffield.ac.uk}
\and Department of Mechanical Engineering\\ ORT Braude College of Engineering, Karmiel, Israel\\ \email{shaulsal@braude.ac.il}}

\maketitle

\begin{abstract}
Abstract here
\end{abstract}

\section{Introduction}
\todo{Just something to work with. Obviously this needs to be rewritten}A general single-objective robust optimisation problem can be formulated as:
\begin{align}
\label{eq:rev:robust}
\min_{\vx\in\Omega} S=\vf\brr{\vx,\vU}.
\end{align}
Here, $\vU$ is a vector of random variables that includes all the uncertainties associated with the optimisation problem.
These uncertainties may be an outcome of manufacturing tolerances, a noisy environment, evaluation inaccuracies, and so on.
A single scenario of the variate $\vU$ is denoted as $\vu$.
Since uncertainties are involved, the scalar objective $S$ is also a random variate, where every scenario of the uncertainties, $\vu$, is associated with an objective value $s$.

In a robust optimisation scheme, the random objective value is replaced with a robustness criterion, denoted by the indicator $\I{S}$.
Several criteria are commonly used in the literature, which can be broadly categorised into three main approaches:
\begin{enumerate}
%
\item \textbf{Worst-Case Scenario.} The worst objective vector, considering a bounded domain in the neighbourhood of the nominal values of the uncertain variables.
%
\item \textbf{Aggregated Value.} An integral measure of robustness that amalgamates the possible values of the uncertain variables (e.g. mean value or variance).
%
\item \textbf{Threshold Probability.} The probability for the objective function to be better than a defined threshold.
\end{enumerate}

In this framework the third approach, suggested by Beyer and Sendhof~\cite{Beyer2007}, is used.
A threshold $q$ is considered as a satisfying performance for the objective value $s$.
When $s$ is uncertain, denoted by the random variable $S$, the probability for $S$ to satisfy the threshold level can be seen as a confidence level $c$.
For a minimization problem this can be written as:
\begin{align}
\label{eq:confidence}
	c\brr{S,q}=\text{Pr}\brr{S<q}.
\end{align}
Equation~\eqref{eq:confidence} can be used for two different robustness indicators:
\begin{enumerate}
	\item Minimization of the threshold $q$ for a pre-defined confidence level $c$, denoted as $\Ic{\bullet}$.
		This is useful when there is no specific target for performance, but the confidence in the resulting performance can be specified.
		The preferred solution is the one that guarantees the best performance with the specified confidence.
	\item Maximization of the confidence level $c$ for a given threshold $q$, denoted as $\Iq{\bullet}$.
		This measure can be used when the target for performance is known, and the emphasis is on meeting this target, rather than performing as well as possible.
\end{enumerate}


The algorithm focuses on the following stochastic unconstrained multiobjective optimization problem (MOP):
\begin{align}
\label{eq:mop}
	\min_{\vx\in\Omega} \vZ=\vf\brr{\vx, \vU},
\end{align}
where $\vx$ is a vector of $n_x$ decision variables in a feasible domain $\Omega$, $\vZ$ is a multivariate random vector of $n_z$ performance criteria and $\vf$ is a set of functions mapping from decision-space to objective-space:
\begin{align}
	\vf: \mathbb{R}^{n_x} \rightarrow \mathbb{R}^{n_z}.
\end{align}
Due to uncertainties over the problem parameters or the mapping functions themselves, every evaluation of the same decision vector may result in a different realisation of the objective vector $\vz$.
If some of the objectives are in conflict with one another, the solution to~\eqref{eq:mop} is expected to be a set of decision vectors, offering different trade-offs between the objectives.

\section{sParEGO}
The main idea is that the uncertain distribution in objective space of every candidate solution is not quantified through Monte-Carlo sampling\footnote{or other uncertainty quantification methods such as polynomial Chaos}, but every solution is evaluated once, and the distribution is approximated based on the performance of nearby solutions.

A pseudo-code of sParEGO is presented in Algorithm~\ref{alg:sParEGO}.
Its stages are explained in details at the remaining of this section.

\begin{algorithm}
\caption{\textsc{sParEGO} Pseudo-code}
\label{alg:sParEGO}
\begin{algorithmic}[1]
	\Statex \textbf{Parameters:} initial set size $n_\text{init}$, robustness criterion $I$, \Comment{Section~\ref{subsec:Parameters}}
	\Statex \hspace{20mm} evaluation functions $\vf$, neighbourhood distance $\delta$,
	\Statex \hspace{20mm} number of reference direction vectors $n_d$
	\State $\DSet \leftarrow$ \textsc{SimplexLattice}($n_d$) \Comment{Secton~\ref{subsec:Simplex Lattice}}
	\State $\XSet \leftarrow$ initialise a set with neighbourhoods of solutions \Comment{Section~\ref{subsec:initialisation}}
	\State $\ZSet \leftarrow \vf\brr{\XSet}$ \Comment{evaluate the initial set}
	\While{stopping criteria not satisfied}
		\State Shuffle the set $\DSet$
		\ForAll{$\vd\in\DSet$}
			\ForAll{$\vx^i\in\XSet$}
				\State update neighbourhood $\NSet^i$ \Comment{Section~\ref{subsec:Uncertainty quantification}}
			\EndFor
			\State update ideal and nadir vectors
			\State $\SSet \leftarrow$ calculate scalar fitness value of all solutions \Comment{Section~\ref{subsec:Scalarising}}
			\State $\ISet \leftarrow \emptyset$
			\ForAll{$\vx^i\in\XSet$}
				\State approximate the distribution of $S^i$ \Comment{Section~\ref{subsec:Uncertainty quantification}}
				\State calculate robustness indicator $\I{S^i}$ \Comment{Section~\ref{subsec:robustness indicators}}
				\State $\ISet \leftarrow \ISet \cup \I{S^i}$
			\EndFor
			\State $model \leftarrow$ fit a Kriging model to the indicator values $\ISet$ \Comment{Section~\ref{subsec:Kriging}}
			\State $\vx^\text{new} \leftarrow$ maximize the expected improvement based on $model$
			\State $\vx^\text{pert} \leftarrow$ add a neighbour to $\vx^\text{new}$ \Comment{Section~\ref{subsec:Kriging}} 
			\State $\XSet \leftarrow \XSet \cup \brf{\vx^\text{new}, \vx^\text{pert}}$
			\State $\ZSet \leftarrow \ZSet \cup \brf{\vf\brr{\vx^\text{new}}, \vf\brr{\vx^\text{pert}}}$ \Comment{evaluate the new solutions}
		\EndFor
	\EndWhile
%\Statex
\end{algorithmic}
\end{algorithm}

\subsection{Parameters}
\label{subsec:Parameters}
\begin{itemize}
\item $\delta$ -- Neighbourhood distance.
Maximum Euclidean distance between solutions, in normalised decision space, to be considered as neighbours.
Default value is $0.1 \sqrt{n_x}$.
\item $n_\text{init}$ -- Size of initial population.
\item $n_\text{indp}$ -- Portion solutions in the initial population that are independently generated.
Must be smaller or equal to $n_\text{init}/2$.
\item $\delta_\text{pert}$ -- maximum distance between newly generated solutions.
Default value is $\delta / 2$.
\item $n_d$ -- Number of sub-problems for decomposition (number of reference direction vectors).
\item Scalarisation function. The norm $p$ when using weighted $Lp$ scalarisation.
\item $c$ -- Desired level of confidence in the performance of the solutions.
\item $\alpha$ -- Level of confidence in parameters of the sampled distribution. Default is $c$.
\item $\beta$ -- Bandwidth of the Kriging model. Defines the intensity of spatial correlation between solutions in the model. Default is $0.1 \delta$.
\item $n_\text{max}$ -- Maximum number of solutions to construct the Kriging model.
\end{itemize}

\subsection{Decomposition}
\label{subsec:Simplex Lattice}

Similar to ParEGO, the multiobjective problem is decomposed into multiple single-objective problems, where at every iteration, a focus is given for a different ratio between the objectives.
However, there are two differences:
\begin{enumerate}
\item Instead of generating a set of uniformly distributed weight vectors, a set of uniformly distributed reference vectors in objective space (direction vectors) is used.
A weight vector is computed to every direction vector according to the Generalised Decomposition framework~\cite{giagkiozis2014generalized}.
\item The algorithm cycles through all direction vectors before revisiting each direction.
At the beginning of every cycle the order of the weights is shuffled.
\end{enumerate}

The decision variables and objectives are likely to be incommensurable, and therefore both decision-space and objective-space are normalised to non-dimensional units in the following manner:
\begin{align}
	\tilde{x}_i &= \frac{x_i - x^l_i}{x^u_i - x^l_i} , &i=1,\ldots,n_x,\\
	\tilde{z}_j &= \frac{z_j - z^*_j}{z^n_j - z^*_j} , &j=1,\ldots,n_z,
\end{align}
where $x^u_i$ and $x^l_i$ are the upper and lower boundaries of the $i^\text{th}$ decision variable, $z^n_j$ and $z^*_j$ are the $j^\text{th}$ components of the known nadir and ideal vectors\footnote{The ideal vector is composed of the best value of each objective.
The nadir vector is composed of the worst value of each objective amongst all Pareto optimal solutions.
Both vectors are estimated according to the current available information concerning the objective-space.}, and the tilde accent represents a normalised, dimensionless variable.
The normalised values are used for all operations within the algorithm.
Before a candidate design is evaluated, it is re-scaled to the natural dimensions.

\subsubsection{Set of Reference Direction Vectors}
The multiobjective optimization problem is decomposed to $n$ sub-problems as described in Section~\ref{subsec:Scalarising}. Each sub-problem $i$ correspond to a reference direction vector $\vd_i$ from a set $\DSet$. In the case where no preference information exists, the set is constructed using a Simplex Lattice design~\cite{Scheffe1958Experiments}:
\begin{align}
	\label{eq:SimplexLattice}
	\begin{split}
		\DSet = \Biggl\lbrace\vd=\brs{d_1,\ldots,d_{n_z}}\quad \vert \quad &\sum_{j=1}^{n_z} d_j = 1 \quad \wedge \\
		& \forall j, \quad d_j = \frac{l}{h}, l\in\brf{0,\ldots,h} \Biggr\rbrace.
	\end{split}
\end{align}
The number of direction vectors, $n_d$, is determined by the dimensionality of the objective space, and the constant $h$: $\vert\DSet\vert = n_d = \binom{h+n_z-1}{n_z-1}$. The direction vectors are evenly distributed along the hyperplane $\sum d_j = 1$.

\subsection{Initialisation}
\label{subsec:initialisation}
To provide good initial coverage of potential designs, a space filling design (Latin hypercube) is used for the initial set.
However, the algorithm requires all solutions to reside within a distance of $\delta$ from other solutions (at least one).
To accomplish this, only $n_\text{indp} \leq n_\text{init}/2$ solutions are generated by Latin hypercube.
The rest of the solutions are generated as follows:
\begin{enumerate}
\item For every existing solution, another solution is randomly generated within a hypersphere with a radius of $\delta_\text{pert}$.
\item The rest of the solutions are generated by randomly selecting an existing solution and creating a nearby solution within a distance of $\delta_\text{pert}$.
\end{enumerate}
The first stage enforces that every solution has at least one neighbour.
The second stage seeds the initial population with neighbourhoods of different sizes.

\subsection{Scalarisation}
\label{subsec:Scalarising}
The scalarising function used in this framework is the popular \emph{weighted $L_p$}~\cite{koski1987norm,Marler2004Survey}:
\begin{align}
\label{eq:weighted Lp}
	s = \brr{\sum_{i=1}^{n_z} \brr{w_i\cdot \brr{z_i-z_i^*}}^p}^{1/p},
\end{align}
where $\mathbf{z^*}$ is a reference vector in objective space, typically the \emph{ideal} vector, which is composed of the minima of each individual objective.
For a given weighting vector $\vw$, a comparison between two vectors $\vz^{A}$ and $\vz^{B}$ depends on the choice of norm $p$. \todo{can refer to Naveed's EMO17 paper for the choice of norm}

For a given direction vector $\vd$ there is a corresponding weighting vector that minimizes the scalarising function \cite{giagkiozis2014generalized}. The optimal weighting vector $\mathbf{w}$ for the scalarising function in~\eqref{eq:weighted Lp} is defined as:
\begin{align}
\begin{split}
	&\mathbf{w}=\brs{w_1,\ldots,w_{n_z}},\\
	&w_i=a\cdot\brr{d_i+\epsilon}^{-1} \quad , \quad i=1,\ldots,n_z,\\
	&\sum_{i=1}^{n_z} w_i=1,
\end{split}
\end{align}
where $\epsilon$ is a small number to prevent division by zero, and $a$ is a normalisation factor.



\subsection{Uncertainty quantification}
\label{subsec:Uncertainty quantification}
The most important difference between sParEGO and ParEGO is that the former assumes that the outcome of an evaluation function is a realization of a random variate.
Therefore, the scalarised function value cannot be used directly to construct the surrogate model, and a utility indicator value is used instead.
For every direction vector $\vd$, the surrogate model is constructed to search for a design that will optimize a given robustness indicator (described later in Section~\ref{subsec:robustness indicators}).
The guiding principle is to avoid Monte Carlo methods that repeatedly sample every candidate design to assess its statistical properties in objective-space.
Instead, these properties (specifically, measures of central tendency and dispersion) are approximated from the available information of other candidate design evaluations.

\subsubsection{Approximation of the central tendency}
The stochasticity of the problem might originate from a variaety of sources, including variations in decision-space.
For this type of uncertainty, two designs with similar nominal values can be identical when realised.
Therefore, the performance of a candidate design should be calculated from the performance of neighbouring designs as well.
A distance\footnote{Distance is measured in decision-space by the Euclidean norm.} in normalised decision space $\delta$ is defined, such that two solutions $\vx^i$ and $\vx^j$ are considered as neighbours if:
\begin{align}
	\norm{\vx^i-\vx^j}{2}\leq\delta.
\end{align}
For a solution $\vx^i$, the statistical properties of the scalar fitness are approximated from the neighbouring solutions as follows:
First, the neighbourhood $\NSet^i$ of the solution is defined\footnote{Note that according to~\eqref{eq:neighbourhood}, $\vx^i$ is included in the neighbourhood $\NSet^i$.}:
\begin{align}
	\label{eq:neighbourhood}
	\NSet^i=\brf{\vx^j\in \XSet \,\vert \,\norm{\vx^i-\vx^j}{2}\leq\delta}.
\end{align}
Next, the approximated mean function value, $\mu_s$, is derived from the neighbourhood.
Members that are closer to $\vx^i$ are given a larger weight, denoted as $v$ in Equation~\eqref{eq:neighbourhood weights}, in approximating its properties.
Since the weight of most solutions in the neighbourhood is smaller than 1, the overall "neighbourhood size" $N^i$ is smaller than $\brabs{\NSet^i}$:
\begin{align}
	\label{eq:neighbourhood weights}
	v^j &= \frac{\delta-\norm{\vx^i-\vx^j}{2}}{\delta}, \quad  \forall \vx^j\in\NSet^i,\\
	\label{eq:neighbourhood size}
	N^i &= \sum_{\vx^j\in\NSet^i} v^j,\\
	\label{eq:neighbourhood mean}
	\mu^i_s &= \left. \sum_{\vx^j\in\NSet^i} v^j s^j \middle/ N^i \right. .
\end{align}

In Figure~\ref{subfig:indicator_estimation_1} the scalar fitness values of four candidate solutions (with a single decision variable) are depicted as grey dots.
The values of the expected mean $\mu_s\brr{x}$ are depicted as black dots.

\subsubsection{Approximation of the dispersion}
Once the expected mean is known, the expected value for the variance is calculated:
\begin{align}
	\label{eq:neighbourhood variance}	\sigma^{2,i}_s = \left. \sum_{\vx^j\in\NSet^i} v^j \brr{s^j - \mu^i_s}^2 \middle/ N^i \right. .
\end{align}
Since the variance in~\eqref{eq:neighbourhood variance} is calculated from a sample (possibly small), the standard error of the sample needs to be taken into account. \todo{consider leaving this part out}
This can be calculated from a chi-squared distribution with $N^i-1$ degrees of freedom.
For a confidence level $\alpha$ for the variance, the upper bound of the confidence interval is:
\begin{align}
	\label{eq:variance upper bound} \sigma^{\prime 2,i}_s = \frac{\sigma^{2,i}_s\brr{N^i-1}} {X^2_{1-\alpha/2}},
\end{align}
where $X^2_{1-\alpha/2}$ is the $1-\alpha/2$ percentile of a chi-squared distribution with $N-1$ degrees of freedom.
The intervals of $\mu_s\pm\sigma_s$ and $\mu_s\pm\sigma'_s$ are depicted in Figure~\ref{subfig:indicator_estimation_b} in red and blue, respectively.

\begin{figure}% 3D graphical demonstration
\centering
\subfigure[Estimation of the mean value and the variance from the neighbourhood]{
	\label{subfig:indicator_estimation_1}
	\def\svgwidth{0.45\textwidth}
	\import{figures/}{indicator_estimation_1.pdf_tex}
}
\hspace{2mm}
\subfigure[Assuming a normal distribution for $S$ according to $\mu_s$ and $\sigma_s$. The shaded area represents the a confidence of $80\%$.]{
	\label{subfig:indicator_estimation_2}
	\def\svgwidth{0.45\textwidth}
	\import{figures/}{indicator_estimation_2.pdf_tex}
}
\caption{\subref{subfig:indicator_estimation_a}-\subref{subfig:indicator_estimation_c} Approximation of the statistical properties. \subref{subfig:indicator_estimation_d} Calculation of robustness indicator $\Iq{\bullet}$.}
\label{fig:indicator_estimation}
%% label for entire figure
\end{figure}

\subsection{Estimating the robustness indicator value}
\label{subsec:robustness indicators}

Once the mean and variance of the scalar fitness function have been estimated for every solution in the set, the random variable $S\brr{\vx}$ is assumed to follow a normal distribution.
Given a desired confidence level $c$, the robustness indicator $\Ic{S}$ can be calculated.
An example for $\Ic{S}$ is given in Figure~\ref{subfig:indicator_estimation_2}.
The indicator value $\Ic{S}$ is considered as the solution's fitness at the current iteration.

\subsection{Fitting a Surrogate Model to the Fitness}
\label{subsec:Kriging}
Now that every solution is associated with a scalar fitness value, the algorithm proceeds in a similar fashion to EGO and ParEGO \cite{Jones1998Efficient,knowles2005multiobjective}.
A surrogate model is fitted to the fitness values, and the expected improvement function is constructed from the model and its prediction error.

For some \todo{replace ``some reasons'' with the actual reasons (if you know them)} reasons, instead of the DACE model used in EGO, here we constructed a Kriging model with a power variogram.
The model's bandwidth, $\beta$, affects the spatial correlation between samples.
It is advisable to set the bandwidth according the neighbourhood distance, $\delta$.
A relation of $\beta=0.1\delta$ has given good results on a set of preliminary tests.

Above a certain size (approximately 100 solutions), the Kriging model becomes prohibitively expensive to construct.
When the number of evaluated solutions exceed this size, a subset of size $n_\text{max}$ is is chosen according to Algorithm~\ref{alg:filtration}.

\begin{algorithm}
\caption{Choosing a Subset to Construct the Surrogate Model}
\label{alg:filtration}
\begin{algorithmic}[1]
	\Require population set $\XSet$, subset size $n_\text{max}$, current direction vector $\vd$
	\Ensure a subset $\XSet'$ of size $n_\text{max}$
	\State $\XSet' \leftarrow n_\text{max}/2$ solutions from $\XSet$ with the best indicator value
	\State $\XSet'' \leftarrow \XSet \setminus \XSet'$
	\ForAll{$\vx \in \XSet''$}
		\State $\hat{\vz}\brr{\vx} \leftarrow$ normalise $\vz\brr{\vx}$ and project it on the $k-1$ simplex
		\State $\Delta\brr{\vx, \vd} \leftarrow \norm{\hat{\vz}\brr{\vx} - \vd}{2}$
	\EndFor
	\State $\XSet' \leftarrow \XSet' \cup n_\text{max}/2$ solutions from $\XSet''$ with the smallest $\Delta$ distance
\end{algorithmic}
\end{algorithm}

\subsection{Optimization of the Expected Improvement}
A single objective global optimization algorithm is used to maximize the expected improvement.
Here we used an evolutionary algorithm ACROMUSE \cite{Ginley2011}.
Once a solution $\vx^\text{new}$ that maximizes the expected improvement is identified, it is added to the population together with a neighbouring solution $\vx^\text{pert}$, generated in a similar fashion to the initial solutions described in Section~\ref{subsec:initialisation}.

\section{Other aspects of the framework}
\label{sec:Other aspects}
\subsection{Attitude to risk}
\label{subsec:risk}
\todo{maybe we want to include this in the conclusions?}
In robust design optimization, there is an inherent trade-off between the solution's expected performance and the confidence in achieving it under extreme scenarios of the uncertainties involved. sParEGO provides four parameters to express the decision maker attitude to risk:
\begin{itemize}
	\item When the robustness metric $\Iq{\bullet}$ is used, the confidence level $c$ can be specified.
	\item When estimating the variance upper bound $\sigma^{\prime 2}_s$  in Equation~\eqref{eq:variance upper bound}, the confidence level $\alpha$ can be specified.
	\item The value used for approximating the variance of isolated solutions $\sigma^{\prime\prime 2}_s$ in Figure~\ref{subfig:indicator_estimation_c}, represents a risk averse attitude. A less conservative approach would be not to consider a smaller value than the upper bound of the model prediction interval.
	\item The value $\delta$--the maximum distance between solutions to be considered as neighbours--affects the variance of solutions and the convergence rate.
		For smooth and continuous functions, a tight neighbourhood is likely to result in smaller variance, but also uses less information from other solutions, which reduces convergence rate.
\end{itemize}

\begin{figure}
\centering
\def\svgwidth{0.45\textwidth}
\import{figures/}{convergence_hv.pdf_tex}
\caption{WE CAN USE THIS FIGURE TO EXPLAIN THE COMPARISON BETWEEN ParEGO AND sParEGO. OR DID WE USE THE PERCENTILES OF THE MARGINAL DISTRIBUTIONS?}
\label{fig:convergence}
\end{figure}

\section{Conclusion}
\label{sec:conclusion}

\subsection{Framework limitations and risks}
This section summarises the assumptions made within the framework described in this document, and the risks associated with these assumptions.
\begin{itemize}
	\item \emph{The landscape is well behaved (i.e., smooth, continuous).} The uncertainty distributions are approximated according to available information for other candidate solutions. The underlying assumption for approximating in this way is that similar solutions have similar performance. If the functions are highly ragged and discontinuous, the surrogate models cannot accurately predict their behaviour.
	\item \emph{The dimensionality is small to medium.} The search is conducted on a surrogate model fitted to the existing evaluated solutions. The DACE model used in this framework typically produces good estimates for problems with up to 20 design variables.
	\item \emph{The problem formulation is appropriately elicited.} The outcomes from any use of the framework can only be as good as the information provided to it. Methodologies for robustly eliciting the problem formulation lie outside the scope of the project.
\end{itemize}


\bibliographystyle{splncs}
\bibliography{sparegoReferences}

\end{document}